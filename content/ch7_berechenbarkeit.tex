%!TEX root = ../doc.tex

\section{Berechenbarkeit}
%%!TEX root = ../../doc.tex

\subsection{Einführung}


%%!TEX root = ../../doc.tex

\subsection{Turing berechenbare Funktion}


%%!TEX root = ../../doc.tex

\subsection{LOOP}


%%!TEX root = ../../doc.tex

\subsection{WHILE}


%%!TEX root = ../../doc.tex

\subsection{Ackermann Funktion}


%%!TEX root = ../../doc.tex

\subsection{(Semi-) Entscheidbarkeit}


%%!TEX root = ../../doc.tex

\subsection{Reduktion}


%%!TEX root = ../../doc.tex

\subsection{Halteproblem}


%%!TEX root = ../../doc.tex

\subsection{Satz von Rice}



\subsection{LOOP}
Variablen (\(xZ\)), Konstanten, Trennzeichen (\(;\)),Zuweisung (\(=\)), Operationszeichen (\(+,-\)), Keywords (\(LOOP, DO, END\))\\
Ausgabe aus \(x_0\)\\
``LOOP x DO P END''
\paragraph{Bedingung - if}
xi = 1 - xi;\\
xi = 1- xi;\\
LOOP xi DO\\
	P\\
END\\
= \\
IF xi > 0 THEN P\\

\subsection{WHILE}
wie LOOP, aber zusätzlich Keyword (\(WHILE\))\\
``WHILE xi > 0 DO ... END''
