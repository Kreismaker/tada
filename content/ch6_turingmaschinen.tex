%!TEX root = ../doc.tex

\section{Turingmaschinen}
%%!TEX root = ../../doc.tex

\subsection{Einführung}


%%!TEX root = ../../doc.tex

\subsection{Konfiguration und Berechnung}


%%!TEX root = ../../doc.tex

\subsection{Sprache}


%%!TEX root = ../../doc.tex

\subsection{Rechnen}


%%!TEX root = ../../doc.tex

\subsection{Erweiterungen}


%%!TEX root = ../../doc.tex

\subsection{Beschränkungen}


%%!TEX root = ../../doc.tex

\subsection{Chomsky Hierarchie}



\subsection{Turingmaschine - Formal}
\[
M=(Q,\Sigma,\Gamma,\delta,q_0,\sqcup,F)
\]
\begin{tabular}{lll}
	\(Q\)		& endliche Menge von Zuständen		& \(Q=\{q_0,q_1,...,q_n\}\)\\
	\(\Sigma\)	& Eingabealphabet					& \(\Sigma = \{a_1,a_2,...,a_m\}\)\\
	\(\Gamma\)	& Bandalphabet						& \(\Sigma \subset \Gamma\)\\
	\(\delta\)	& Übergangsfunktion					& \(\delta:\)\\
	\(q_0\)		& Startzustand						& \(q_0 \in Q\)\\
	\(\sqcup\)	& Leerzeichen						& \(\sqcup \in \Gamma \land \sqcup \notin \Sigma\)\\
	\(F\)		& Menge der akzeptierenden Zust.	& \(F \subseteq Q\)\\
\end{tabular}

\paragraph{Übergang}
\[
\delta(q,X_i)=(p,Y,R)
\]

\paragraph{Berechnungsschritt}
\[
\text{Bandinhalt}q\text{Bandinhalt}
\]