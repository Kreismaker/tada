%!TEX root = ../doc.tex

\section{Kelleratomaten}
%%!TEX root = ../../doc.tex

\subsection{Deterministik}


\subsection{Deterministischer Kellerautomat - Formal}
\[
KA = (Q, \Sigma, \Gamma,\delta, q_0, \$, F)
\]
\begin{tabular}{lll}
	\(Q\)		& endliche Menge von Zuständen	& \(Q=\{q_0,q_1,...,q_n\}\)\\
	\(\Sigma\)	& Eingabealphabet				& \(\Sigma = \{a_1,a_2,...,a_m\}\)\\
	\(\Gamma\)	& Kelleralphabet				& \(\Gamma = \{a_1,a_2,...,a_m\}\)\\
	\(\delta\)	& Übergangsfunktion				& \(\delta: Q \times (\Sigma \cup \epsilon) \times \Gamma \rightarrow Q \times \Gamma^*\)\\
	\(q_0\)		& Startzustand					& \(q_0 \in Q\)\\
	\(\$\)		& Anfangssysmbol des Kellers	& \(\$ \in \Gamma\)\\
	\(F\)		& Menge akzeptierende Zustände	& \(F \subseteq Q\) \\
\end{tabular}

\paragraph{Beispiel Übergangsfunktion}
\[
	\delta (q_0,a_1,a_2) = (q_0,a_1 a_1)
\]

\subsubsection{Berechnungsschritte}
\[
	?
\]

\subsection{Nichtdetermininismus}
Gleich wie deterministischer KA, aber
\[
\delta: Q \times (\Sigma \cup \epsilon) \times \Gamma \rightarrow \mathcal{P}(Q \times \Gamma^*)
\]